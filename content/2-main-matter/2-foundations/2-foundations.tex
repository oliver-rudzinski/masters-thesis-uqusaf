% - FOUNDATIONS ------------------------------------------------------------
\chapter{Foundations} \label{ch:foundations}
This chapter aims at providing an overview on the necessary background information and technical foundations required to perform the desired research. It is expected that these foundations will be elaborated in the actual thesis as certain examined aspects of the research might require additional background. The design of this chapter is not meant to be exhaustive to each area but shall only provide the information that is strictly required for the sake of this work.

\section{Information Security} \label{sec:foundations-security}

The U.S. \acf{nist} defines information security as ``the protection of information and information systems from unauthorized access, use, disclosure, disruption, modification, or destruction in order to ensure confidentiality, integrity, and availability`` \cite{nieles_introduction_2017}. The last three characteristics can be further defined:
		
	\begin{description}
		\item[Confidentiality] Unauthorized entities are restricted from access and disclosure of information for the purpose of protecting personal privacy and proprietary information \cite{nieles_introduction_2017}.
		\item[Integrity] Improper modification or destruction of information, including the compromitation of data non-repudiation and authenticity, is protected against. Furthermore, \textit{data integrity} means that data has not been improperly altered, whereas \textit{system integrity} goes beyond unautorized manipulation, defining system function quality when performing as intended and unimpaired \cite{nieles_introduction_2017}.
		\item[Availability] Accessibility and usability of information is reliable and timely \cite{nieles_introduction_2017}. This also includes the prevention of intended, unauthorized withholding of data \cite[p. 36]{fischer-hubner_it-security_2001}.
	\end{description}
		
	This \textit{CIA triad} can be extended by additional characteristics, e.g. \textit{accountability} (user actions can be traced back to them), \textit{functionality} (system's behavior is as intended and expected) and \textit{reliability} (system always performs under equal conditions) \cite[p. 36 sq.]{fischer-hubner_it-security_2001}.
 
\subsection{Distinction from Cyber Security} \label{subs:foundations-security-distinction}

The terms \textit{information security} and \textit{cyber security} (i.a.), are often used interchangeably in literature. However, this work requires a specific distinction between those terms. As defined by the \ac{nist} Cybersecurity Framework, cyber security means ``the ability to protect or defend the use of cyberspace from cyber attacks`` \cite{noauthor_framework_2018}. By that, cyber security is a disciplinary subset of information security that especially considers the electronic protection of data and resources within its technical perimeter. Consequently, this excludes activities like physical access or compliance control that are inherent information security disciplines. Nevertheless, both areas focus on achieving the goal of the information security definition by ensuring the application of the CIA tiad.

\subsection{Security Design} \label{subs:foundations-security-design}
As motivated in \autoref{ch:intro}, cyberattacks are still on the rise, which leads to assume that the current state of the art of security design and implementation cannot systematically exclude security flaws. This has already been recognized by Saltzer and Schröder (1975) who summarized general security design guidelines that address general security problems during the design of information systems \cite{saltzer_protection_1975}. They include best practices that aim at considering the CIA triad from the software perspective.%Their Security Design Principles include the following:

%\begin{description}
%	\item[Economy of mechanism] The design of security measures embodied in both hardware and software should be as simple and small as possible \cite{saltzer_protection_1975}.
%	\item[Fail-safe defaults] Access decisions should be based on permission rather than exclusion \cite{saltzer_protection_1975}.
%	\item[Complete mediation] Every access to resources must be checked against the access control mechanism \cite{saltzer_protection_1975}. In turn, no access should be granted without authorization.
%	\item[Open design] The design of a security mechanism should be open rather than secret \cite{saltzer_protection_1975}. The absence of that rule is also referred to as \textit{security by obscurity}.
%	\item[Separation of privilege] Multiple privilege attributes are required to access a restricted resource \cite{saltzer_protection_1975}.  
%	\item[Least privilege] Every process and every user of the system should operate using the least set of privileges necessary to perform the task \cite{saltzer_protection_1975}.
%	\item[Least common mechanism] Design should minimize the functions shared by different users \cite{saltzer_protection_1975}.
%	\item[Psychological acceptability] Security mechanisms should not interfere unduly with the work of users \cite{saltzer_protection_1975}. Later, it will be shown that this principle laid the foundation for usable security research.
%\end{description}


It appears to support security design aspects during the actual design and implementation. There exist similar design frameworks in that domain, e.g., the \ac{owasp} Security By Design Principles. %They also include aspects of less trust in terms of third-party services as well as the necessity to mitigate security vulnerabilities at their core.

Finally, there exist distinct security assessment and improvement frameworks for information systems that are already in place, such as the \ac{nist} Cybersecurity Framework. It defines a process model for the evaluation of the current state, a tiered readiness assessment, as well as transition profiles to reach a desired state of cybersecurity \cite{noauthor_framework_2018}.%, starting with 

%\begin{itemize}
	%\item the identification of security threats and vulerabilities,
	%\item the development of protection safeguards,
	%\item the detection of security incidents,
	%\item the development of response mechanisms to these incidents, as well as the
	%\item recovery from these events after the fact \cite{noauthor_framework_2018}.
%\end{itemize}

%Furthermore, the \ac{nist} framework also works with a readiness evaluation that assesses the degree of completeness to its security processes. Additionally, the framework defines profiles that support the transition from current, potentially undesirable security measures, to a desired target state \cite{noauthor_framework_2018}.


\section{Usability of Information Systems} \label{sec:foundations-usability}

Information system usability is part of ergonomics of human-system interaction, standardized by the \ac{iso} \cite{noauthor_ergonomics_2020}. It defines \textit{usability} as the ``extent to which a system, product or service can be used by specified users to achieve specified goals with effectiveness, efficiency and satisfaction in a specified context of use`` \cite{noauthor_ergonomics_2020}.

The interaction between the user and the system is done through a \acf{ui}, defined by \ac{iso} as the ``set of all the components of an interactive system that provide information and controls for the user to accomplish specific tasks with the interactive system`` \cite{noauthor_ergonomics_2020}.

While usability design and evaluation are often applied to \acp{ui}, these processes could also apply to less end-user-focussed interfaces, e.g., \acp{api} or \acp{cli}.

\subsection{Usability Design Guidelines} \label{subs:foundations-usability-design}

The process of providing \textit{good} system usability to users is aided by several guidelines and frameworks: The \ac{iso} defines the \ac{hcd} process, which describes an ``approach to systems design and development that aims to make interactive systems more usable by focusing on the use of the system and applying human factors/ergonomics and usability knowledge and techniques`` \cite{noauthor_ergonomics_2019}.

\ac{hcd} is an iterative approach and includes the understanding and requirement of the context, as well as the cooperative development and evaluation of design solutions \cite{noauthor_ergonomics_2019}. The latter steps are aided by potential users of the system under design.

\ac{hcd} is particularized through further guidelines developed by notable researchers that define general principles on usability design. These include, but are not limited to, Shneiderman's Eight Golden Rules of Interface Design \cite{shneiderman_designing_1987}, Norman's Seven Design Principles \cite{norman_design_2013}, and Nielsen's Ten Usability Heuristics for User Interface Design \cite{nielsen_heuristic_1990}. All of those frameworks follow a similar structure in that they specifically ask for the existence or absence of characteristics within the interface, e.g., visibility of things, consistency of design, etc.

\subsection{Usability Evaluation} \label{subs:foundations-usability-evaluation}

The evaluation of usability is often referred to \textit{usability testing} \cite{moran_usability_2019}. Its goals are to identify problems, uncover opportunities as well as learn about the target user's behavior and preferences \cite{moran_usability_2019}. Elements of usability testing include participants (i.e., realistic users of the system), specific tasks or activities performed by the participant, as well as a facilitator (i.e., moderator and creator of the specific activity) \cite{moran_usability_2019}. After performing the activities, the participants are expected to provide feedback to the moderator, whereas the moderator notes insights to the behavior of the participant and the system.

Usability evaluation can be either qualitative (i.e., focussing on distinct insights and findings with regards to user behavior) or quantitative (i.e., focussing on benchmarks, e.g., task success or elapsed time). They are usually conducted in a remote setting, however there also exists in-person usability testing. Finally, it can be performed in a moderated (i.e., interactive) or unmoderated manner (i.e., no interaction between the participant and the facilitator) \cite{moran_usability_2019}.

The definition of tasks for a usability evaluation highly depends on the specific area under test. However, it can be expected that the usability design guidelines (cf. \autoref{subs:foundations-usability-design}) can be taken into account.

\section{Usable Security} \label{sec:foundation-usable_security}

Bringing it all together, \textit{usable security} deals with the design, implementation, and evaluation of the usability of security features and aspects within information systems \cite{reuter_quarter_2022}. Chronologically, it originates from the already mentioned Security Design Principles by Saltzer and Schröder (1975), specifically the final one considering the ``psychological acceptability`` of the mechanisms \cites{reuter_quarter_2022}{saltzer_protection_1975}. Later, Zurko and Simon (1996) novelly suggested to apply usability aspects of software design to security features \cite{zurko_user-centered_1996}. Shortly after, Adams and Sasse (1999) researched usability flaws in terms of password protection from a user perspective and provided recommendations on remedial measures \cite{adams_users_1999}. Since then, the majority of usable security research is focussed on end-user security in terms of password protection and email encryption \cites{reuter_quarter_2022}{whitten_why_1999}. More recently, this field also started to consider developers and \ac{it} professionals as their usability issues in terms of security can lead to severe security vulnerabilities \cite{chiasson_even_2007-1}.

Other than with security \textit{or} usability evaluation, there do not exist established unified frameworks or guidelines for usable security assessment. Some more recent studies suggest certain approaches that will be discussed in \autoref{ch:related-work}. Other than that, usable security is currently performed by means of individual experiments \cite{garfinkel_usable_2014}.

\section{\acl{cc}} \label{sec:foundations-cloud}

The \ac{nist} defines \ac{cc} as ``a model for enabling ubiquitous, convenient, on-demand network access to a shared pool of configurable computing resources (...) that can be rapidly provisioned and released with minimal management effort (...)`` \cite{mell_nist_2011}. It can be seen as the latest form of infrastructure consolidation after the introduction of client-server computing and the rise of datacenter hardware virtualization \cite[p.\ 1]{sehgal_cloud_2018}.
		
In general, \ac{cc} is described by a vendor-consumer relationship, with the vendor being the provider of the \ac{it} resources, and the consumer being the receiver and user of those resources. Depending on the deployment model of the \ac{cc} infrastructure at hand, the vendor can be 
		
\begin{itemize}
	\item internal to an organization or providing isolated resources exclusively to that organization (i.e., \textit{private cloud}),
	\item external, providing resources to multiple entities (i.e., \textit{public cloud}),
	\item or a mix of both \cite{mell_nist_2011}. 
\end{itemize}  
		
The service level defines the level of abstraction of the resources to the consumer and which party is responsible for which layer of application provisioning and deployment. Typically, this includes \ac{iaas} (vendor provides virtualized infrastructure), \ac{paas} (vendor provides virtualized infrastructure as well as the specific operating system, runtime environment, etc. to that platform), and \ac{saas} (vendor provides infrastructure, platform, and specific software deployment) \cite{mell_nist_2011}. 
		
The \ac{nist} laid out several aspects that further define \ac{cc} characteristics, service, and deployment models.
		
\begin{description}
	\item[On-demand self-service] Provisioning of (more) computing, storage, network resources, etc. is done by the consumer without interaction with the \ac{cc} service provider \cite{mell_nist_2011}. This implies a form of automation on the \ac{cc} vendor's end as well as the existence of control and monitoring interfaces for the consumer \cite[p.\ 51]{sehgal_cloud_2018}. 
	\item[Broad network access] The \ac{cc} vendor's services and capabilities are accessible over the network and can be accessed via mechanisms that are already in place for consuming clients \cites{mell_nist_2011}[p.\ 45]{sehgal_cloud_2018}.
	\item[Resource pooling] Resource pooling allows \ac{cc} vendor's to introduce the concept of \textit{multi-tenancy}, where the physical infrastructure that is made available is virtualized and dynamically assigned to distinct consumers, or \textit{tenants}, based on demand or contractual agreements \cites{mell_nist_2011}[p.\ 45]{sehgal_cloud_2018}. 
		In other words, a \textit{pool} of total resources present is made available to multiple consumers based on their \textit{current} needs and re-evaluated over the course of the usage.
	\item[Rapid elasticity] The consumer can dynamically scale their resource capacity up or down as needed. The consumer is not aware of any global resource limitations which make the \ac{cc} resources appear unlimited \cite{mell_nist_2011}.
	\item[Measured service] The consumed services are quantitatively measured by the \ac{cc} vendor for billing, controlling and optimization purposes. This is based on appropriate forms of metering with regard to the service (e.g., number of terabytes of storage consumed within a month). Thus, billing of utilized services is often done on a pay-per-use basis \cite{mell_nist_2011}. Consequently, the more resources are utilized by the vendor, the higher the cost. This process is transparent to both vendor and consumer, allowing for consumption and billing monitoring, control, and reporting \cite{mell_nist_2011}.
\end{description}
% --------------------------------------------------------------------------